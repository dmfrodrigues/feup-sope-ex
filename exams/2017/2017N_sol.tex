\setcounter{chapter}{16}
\chapter{Exam 2016/17}
{
\renewcommand{\thesubsection}{\thesection\alph{subsection}}
\section{Exercise 1}
\begin{tabular}{p{29mm} | p{122mm}}
    DMA & Allows some hardware subsystems (usually peripherals) to directly access certain principal memory regions. This means the CPU is only required to start or end transactions, and the actual transactions can be performed by the subsystems, freeing CPU time. \\ \hline
    Test\&Set instruction & Hardware instruction that can be used to implement mutual exclusion and mutexes. Sets variable to 1 and returns the previous value atomically, so a critical section can be implemented by first calling Test\&Set for the \emph{mutex} until it returns 0, execute the section and then set the \emph{mutex} to 0. \\ \hline
    Valid/invalid page bit & Essential part of a virtual memory implementation. There is a table that has a valid/invalid bit for each page, which is set to 1 if it is currently loaded in principal memory, so this bit allows to know if a page is loaded in memory (i.e., can access it) or not (i.e., needs to load that page and only then can it be accessed).
\end{tabular}

\section{Exercise 2}
\begin{lstlisting}[language=C]
// Global
init(ready1, 0);
init(ready2, 0);
init(ready3, 0);
\end{lstlisting}
\begin{tabular}{p{49mm} p{49mm} p{49mm}}
    \begin{lstlisting}[language=C]
// P1
sinal(ready1);
wait(ready2);
signal(ready2);
wait(ready3);
signal(ready3);
// A    
    \end{lstlisting} &
    \begin{lstlisting}[language=C]
// P2
sinal(ready2);
wait(ready1);
signal(ready1);
wait(ready3);
signal(ready3);
// B    
    \end{lstlisting} &
    \begin{lstlisting}[language=C]
// P3
sinal(ready3);
wait(ready1);
signal(ready1);
wait(ready2);
signal(ready2);    
// C
    \end{lstlisting}
\end{tabular}

}
